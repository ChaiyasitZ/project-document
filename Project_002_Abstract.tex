%**************************************************************************************
% บทคัดย่อภาษาไทย
%**************************************************************************************
\phantomsection
\addcontentsline{toc}{chapter}{บทคัดย่อภาษาไทย}
\begin{tabularx}{\textwidth}{lX}
ชื่อ                           & : นางสาวสุพาภรณ์ ซิ้มเจริญ         %ชื่อผู้จัดทำ1 
%\\     %เว้นบรรทัด เมื่อมีเพื่อนร่วมโปรเจ็ค
%                             & : นางสาวสิวาลัย จินเจือ            %ชื่อผู้จัดทำ2 
\\
ชื่อปริญญานิพนธ์                  & : \hangindent=0.5em          % มีไว้สำหรับตอนชื่อโปรเจ็คเกิน 1 บรรทัด
                                การเขียนเล่มปริญญานิพนธ์ด้วย LaTeX  %ระบุชื่อโปรเจ็คภาษาไทย
\\
สาขาวิชา                       & : เทคโนโลยีสารสนเทศ % กรณีเป็นนักศึกษาหลักสูตร IT
%สาขาวิชา                       & : เทคโนโลยีสารสนเทศ (ต่อเนื่อง) % กรณีเป็นนักศึกษาหลักสูตร ITI
%สาขาวิชา                       & : วิศวกรรมสารสนเทศและเครือข่าย % กรณีเป็นนักศึกษาหลักสูตร INE และ INET
\\
                             & : มหาวิทยาลัยเทคโนโลยีพระจอมเกล้าพระนครเหนือ
\\
อาจารย์ที่ปรึกษาปริญญานิพนธ์        & : รองศาสตราจารย์ ดร.อนิราช มิ่งขวัญ    %ระบุชื่ออาจารย์ที่ปรึกษา
%\\      %เว้นบรรทัด เมื่อมีที่ปรึกษาร่วม
%ที่ปรึกษาร่วม                    & : ผู้ช่วยศาสตาจารย์ ดร.นิติการ นาคเจือทอง    %ระบุชื่อที่ปรึกษาร่วม (ถ้ามี)
\\
ปีการศึกษา                     & : 2568     %ระบุปีการศึกษา
\end{tabularx}

\vspace{5mm}
\phantomsection
\begin{center}\textbf{บทคัดย่อ}\end{center}
\hspace*{1.5em} %ย่อหน้า
โครงงาน "การจัดทำรูปเล่มปริญญานิพนธ์ด้วย LaTeX" ฉบับนี้ มีวัตถุประสงค์เพื่อนำเสนอแนวทางการปฏิบัติในการจัดทำเอกสารรูปเล่มปริญญานิพนธ์ ตั้งแต่หน้าปกจนถึงภาคผนวก โดยใช้โปรแกรมประมวลผลเอกสาร LaTeX ซึ่งเป็นที่ยอมรับอย่างแพร่หลายในแวดวงวิชาการ ด้วยความสามารถในการจัดการเอกสารที่มีความซับซ้อนได้อย่างมีประสิทธิภาพ และความแม่นยำในการจัดรูปแบบที่เป็นมาตรฐาน จึงถือเป็นเครื่องมือสำคัญสำหรับนักศึกษา ในการอำนวยความสะดวกต่อการดำเนินการจัดทำรูปเล่มปริญญานิพนธ์ให้มีประสิทธิภาพยิ่งขึ้น โดยเนื้อหาจะครอบคลุมองค์ประกอบสำคัญของปริญญานิพนธ์ เริ่มตั้งแต่ บทคัดย่อภาษาไทย บทคัดย่อภาษาอังกฤษ กิตติกรรมประกาศ สารบัญ สารบัญภาพ สารบัญตาราง บทที่ 1 บทนำ บทที่ 2 แนวคิด ทฤษฎี และงานวิจัยที่เกี่ยวข้อง บทที่ 3 วิธีการดำเนินงาน บทที่ 4 ผลการดำเนินงาน บทที่ 5 สรุปผลการดำเนินงาน บรรณานุกรม รวมถึงภาคผนวก นอกจากนี้ ยังให้ความสำคัญกับการแทรกรูปภาพ ตาราง และการจัดการเอกสารอ้างอิงภายในเนื้อหาอย่างถูกต้องและเป็นระบบ

\begin{flushright}
(ปริญญานิพนธ์มีจำนวนทั้งสิ้น \pageref{LastPage} หน้า)

\vfill
\rule{10 cm}{0.4pt} อาจารย์ที่ปรึกษาปริญญานิพนธ์ 
%\\ \vspace{3mm}                        %เว้นกรณีมีที่ปรึกษาร่วม
%\rule{12.6 cm}{0.4pt} ที่ปรึกษาร่วม         % กรณีมีที่ปรึกษาร่วม

\end{flushright}





%**************************************************************************************
% บทคัดย่อภาษาอังกฤษ
%**************************************************************************************
\newpage
\phantomsection
\addcontentsline{toc}{chapter}{บทคัดย่อภาษาอังกฤษ}
\begin{tabularx}{\textwidth}{lX}
Name                    & : Ms.Supaporn Simcharoen  %ชื่อผู้จัดทำ1
%\\      %เว้นบรรทัดเมื่อมีเพื่อนร่วมโปรเจ็ค
%                       & : Ms.Siwalai Chinchua      %ชื่อผู้จัดทำ2
\\
Project Title           & : \hangindent=0.5em % มีไว้สำหรับตอนชื่อโปรเจ็คเกิน 1 บรรทัด
                            Writing a Project Report with LaTeX  %ระบุชื่อโปรเจ็คภาษาอังกฤษ
\\
Major Field             & : Information Technology % กรณีเป็นนักศึกษาหลักสูตร IT
%Major Field             & : Information Technology (Continuing Program) % กรณีเป็นนักศึกษาหลักสูตร ITI
%Major Field             & : Information and Network Engineering % กรณีเป็นนักศึกษาหลักสูตร INE และ INET
\\
                        & : King Mongkut’s University of Technology North Bangkok
\\
Project Advisor         & : Assoc. Prof. Dr.Anirach Mingkhwan   %ระบุชื่ออาจารย์ที่ปรึกษา
%\\      %เว้นบรรทัด เมื่อมีที่ปรึกษาร่วม
%Co-Advisor              & : Asst. Prof. Dr.Nitigan Nakjuatong   %ระบุชื่อที่ปรึกษาร่วม (ถ้ามี)
\\
Academic Year           & : 2025    %ระบุปีการศึกษา
\end{tabularx}

\vspace{5mm}
\begin{center}\textbf{Abstract}\end{center}

\hspace*{1.5em} %ย่อหน้า
This project, "Writing a Project Report with LaTeX," aims to present practical guidelines for preparing a complete thesis document, from the cover page to the appendices. It utilizes the LaTeX document processing system, widely accepted in academic circles due to its efficient handling of complex documents and precise, standardized formatting. Therefore, LaTeX is considered an essential tool for students, facilitating a more effective and streamlined thesis preparation process. The content covers key components of a thesis, beginning with the Thai Abstract, English Abstract, Acknowledgments, Table of Contents, List of Figures, List of Tables, Chapter 1: Introduction, Chapter 2: Concepts, Theories, and Related Research, Chapter 3: Methodology, Chapter 4: Results, Chapter 5: Conclusion, and Bibliography, including Appendices. Additionally, emphasis is placed on the correct and systematic insertion of figures and tables, as well as the management of in-text citations.

\begin{flushright}
(Total \pageref{LastPage} pages)

\vfill

\rule{12 cm}{0.4pt} Project Advisor
%\\ \vspace{3mm}                          %เว้นกรณีมีที่ปรึกษาร่วม
%\rule{12.7 cm}{0.4pt} Co-Advisor         % กรณีมีที่ปรึกษาร่วม

\end{flushright}


%**************************************************************************************
% กิตติกรรมประกาศ
%**************************************************************************************
\newpage
\phantomsection
\addcontentsline{toc}{chapter}{กิตติกรรมประกาศ}
\begin{center}\textbf{กิตติกรรมประกาศ}\end{center}

\hspace*{1.5em}  %ย่อหน้า
ปริญานิพนธ์ "การเขียนเล่มปริญญานิพนธ์ด้วย LaTeX" จะไม่สามารถสำเร็จลุล่วงไปได้ด้วยดีหากไม่ได้รับการช่วยเหลือและความอนุเคราะห์จากบุคคลหลายท่าน ทางผู้จัดทำต้องขอขอบพระคุณ รองศาสตราจารย์ ดร.อนิราช มิ่งขวัญ ซึ่งเป็นอาจารย์ที่ปรึกษา ที่ได้กรุณาให้คำปรึกษา และแนะนำแนวทางในการจัดทำปริญญานิพนธ์ รวมทั้งคณาจารย์ภาควิชาเทคโนโลยีสารสนเทศที่ได้ให้ความรู้กับผู้จัดทำเพื่อนำมาประยุกต์ในการจัดทำปริญญานิพนธ์นี้ 
\\
\hspace*{1.5em}  %ย่อหน้า
ท้ายที่สุดนี้ ทางผู้จัดทำขอน้อมรำลึกพระคุณบิดา มารดา ผู้ซึ่งมีพระคุณอย่างสูงสุดที่ให้ความอุปการะผู้จัดทำมาโดยตลอด รวมทั้งผู้มีพระคุณทุกท่าน คณะผู้จัดทำรู้สึกซาบซึ้งเป็นอย่างยิ่ง จึงใคร่ขอขอบพระคุณเป็นอย่างสูงมา ณ โอกาสนี้ด้วย
\\
\begin{flushright}
\makebox[6cm][l]{สุพาภรณ์ ซิ้มเจริญ}      %ชื่อผู้จัดทำ1
%\\     %เว้นบรรทัด เมื่อมีเพื่อนร่วมโปรเจ็ค
%\makebox[6cm][l]{สิวาลัย จินเจือ}         %ชื่อผู้จัดทำ2

\end{flushright}