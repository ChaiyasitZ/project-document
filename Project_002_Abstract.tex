%**************************************************************************************
% บทคัดย่อภาษาไทย
%**************************************************************************************
\phantomsection
\addcontentsline{toc}{chapter}{บทคัดย่อภาษาไทย}
\begin{tabularx}{\textwidth}{lX}
ชื่อ                           & : นายชัยสิทธิ์ มินทกร         %ชื่อผู้จัดทำ1
%\\     %เว้นบรรทัด เมื่อมีเพื่อนร่วมโปรเจ็ค
%                             & : นางสาวสิวาลัย จินเจือ            %ชื่อผู้จัดทำ2 
\\
ชื่อปริญญานิพนธ์                  & : \hangindent=0.5em          % มีไว้สำหรับตอนชื่อโปรเจ็คเกิน 1 บรรทัด
                                เว็บแอปพลิเคชันเพื่อการคอนฟิกด้วยวิธี NETCONF ผ่าน LLM
\\
สาขาวิชา                       & : วิศวกรรมสารสนเทศและเครือข่าย % กรณีเป็นนักศึกษาหลักสูตร IT
%สาขาวิชา                       & : เทคโนโลยีสารสนเทศ (ต่อเนื่อง) % กรณีเป็นนักศึกษาหลักสูตร ITI
%สาขาวิชา                       & : วิศวกรรมสารสนเทศและเครือข่าย % กรณีเป็นนักศึกษาหลักสูตร INE และ INET
\\
                             & : มหาวิทยาลัยเทคโนโลยีพระจอมเกล้าพระนครเหนือ
\\
อาจารย์ที่ปรึกษาปริญญานิพนธ์        & : อาจารย์ ดร.วัชรชัย คงศิริวัฒนา   %ระบุชื่ออาจารย์ที่ปรึกษา
%\\      %เว้นบรรทัด เมื่อมีที่ปรึกษาร่วม
%ที่ปรึกษาร่วม                    & : ผู้ช่วยศาสตาจารย์ ดร.นิติการ นาคเจือทอง    %ระบุชื่อที่ปรึกษาร่วม (ถ้ามี)
\\
ปีการศึกษา                     & : 2568     %ระบุปีการศึกษา
\end{tabularx}

\vspace{5mm}
\phantomsection
\begin{center}\textbf{บทคัดย่อ}\end{center}
\hspace*{1.5em} %ย่อหน้า
โครงการนี้มีวัตถุประสงค์ในการพัฒนาเว็บแอปพลิเคชันเพื่อการจัดการเครือข่าย ที่ช่วยให้ผู้ดูแลระบบสามารถดำเนินการตั้งค่าอุปกรณ์เครือข่าย Cisco ได้อย่างมีประสิทธิภาพและสะดวกยิ่งขึ้น ระบบที่พัฒนาขึ้นรองรับการกำหนดค่าพื้นฐาน เช่น การตั้งค่า IP Address การกำหนดค่า VLAN และการจัดการ Routing Protocols โดยใช้ Node.js และ Express สำหรับการพัฒนา Backend และ React สำหรับการพัฒนา Frontend พร้อมทั้งเชื่อมต่อกับ MongoDB เพื่อจัดเก็บข้อมูลอุปกรณ์

\hspace*{1.5em} %ย่อหน้า
นอกจากนี้ ระบบยังผสานเทคโนโลยี Large Language Model (LLM) เพื่อสร้างการกำหนดค่าอัตโนมัติจาก Prompt ที่ผู้ใช้กรอก ช่วยลดความซับซ้อนและความผิดพลาดในการกำหนดค่าด้วยตนเอง อีกทั้งยังมีความสามารถในการ สำรองข้อมูล (Backup) และ กู้คืนการตั้งค่า (Rollback) ของอุปกรณ์เพื่อรองรับการจัดการอย่างปลอดภัย

\hspace*{1.5em} %ย่อหน้า
ผลลัพธ์ที่ได้จากโครงการช่วยเพิ่มประสิทธิภาพในการจัดการเครือข่าย ลดระยะเวลาในการกำหนดค่า และเพิ่มความยืดหยุ่นในการดูแลรักษาอุปกรณ์เครือข่าย Cisco ให้มีความทันสมัยและสอดคล้องกับความต้องการขององค์กรในยุคดิจิทัล
\begin{flushright}
(ปริญญานิพนธ์มีจำนวนทั้งสิ้น \pageref{LastPage} หน้า)

\vfill
\rule{10 cm}{0.4pt} อาจารย์ที่ปรึกษาปริญญานิพนธ์ 
%\\ \vspace{3mm}                        %เว้นกรณีมีที่ปรึกษาร่วม
%\rule{12.6 cm}{0.4pt} ที่ปรึกษาร่วม         % กรณีมีที่ปรึกษาร่วม

\end{flushright}





%**************************************************************************************
% บทคัดย่อภาษาอังกฤษ
%**************************************************************************************
\newpage
\phantomsection
\addcontentsline{toc}{chapter}{บทคัดย่อภาษาอังกฤษ}
\begin{tabularx}{\textwidth}{lX}
Name                    & : MR.CHAIYASIT MINTAKORN  %ชื่อผู้จัดทำ1
%\\      %เว้นบรรทัดเมื่อมีเพื่อนร่วมโปรเจ็ค
%                       & : Ms.Siwalai Chinchua      %ชื่อผู้จัดทำ2
\\
Project Title           & : \hangindent=0.5em % มีไว้สำหรับตอนชื่อโปรเจ็คเกิน 1 บรรทัด
                            Web Application for NETCONF Configuration via LLM  %ระบุชื่อโปรเจ็คภาษาอังกฤษ
\\
Major Field             & : Information and Network Engineering % กรณีเป็นนักศึกษาหลักสูตร IT
%Major Field             & : Information Technology (Continuing Program) % กรณีเป็นนักศึกษาหลักสูตร ITI
%Major Field             & : Information and Network Engineering % กรณีเป็นนักศึกษาหลักสูตร INE และ INET
\\
                        & : King Mongkut’s University of Technology North Bangkok
\\
Project Advisor         & : Dr.Watcharachai Kongsiriwattana  %ระบุชื่ออาจารย์ที่ปรึกษา
%\\      %เว้นบรรทัด เมื่อมีที่ปรึกษาร่วม
%Co-Advisor              & : Asst. Prof. Dr.Nitigan Nakjuatong   %ระบุชื่อที่ปรึกษาร่วม (ถ้ามี)
\\
Academic Year           & : 2025    %ระบุปีการศึกษา
\end{tabularx}

\vspace{5mm}
\begin{center}\textbf{Abstract}\end{center}

\hspace*{1.5em} %ย่อหน้า
This project, "Writing a Project Report with LaTeX," aims to present practical guidelines for preparing a complete thesis document, from the cover page to the appendices. It utilizes the LaTeX document processing system, widely accepted in academic circles due to its efficient handling of complex documents and precise, standardized formatting. Therefore, LaTeX is considered an essential tool for students, facilitating a more effective and streamlined thesis preparation process. The content covers key components of a thesis, beginning with the Thai Abstract, English Abstract, Acknowledgments, Table of Contents, List of Figures, List of Tables, Chapter 1: Introduction, Chapter 2: Concepts, Theories, and Related Research, Chapter 3: Methodology, Chapter 4: Results, Chapter 5: Conclusion, and Bibliography, including Appendices. Additionally, emphasis is placed on the correct and systematic insertion of figures and tables, as well as the management of in-text citations.

\begin{flushright}
(Total \pageref{LastPage} pages)

\vfill

\rule{12 cm}{0.4pt} Project Advisor
%\\ \vspace{3mm}                          %เว้นกรณีมีที่ปรึกษาร่วม
%\rule{12.7 cm}{0.4pt} Co-Advisor         % กรณีมีที่ปรึกษาร่วม

\end{flushright}


%**************************************************************************************
% กิตติกรรมประกาศ
%**************************************************************************************
\newpage
\phantomsection
\addcontentsline{toc}{chapter}{กิตติกรรมประกาศ}
\begin{center}\textbf{กิตติกรรมประกาศ}\end{center}

\hspace*{1.5em}  %ย่อหน้า
ปริญานิพนธ์ "เว็บแอปพลิเคชันเพื่อการคอนฟิกด้วยวิธี NETCONF ผ่าน LLM" จะไม่สามารถสำเร็จลุล่วงไปได้ด้วยดีหากไม่ได้รับการช่วยเหลือและความอนุเคราะห์จากบุคคลหลายท่าน ทางผู้จัดทำต้องขอขอบพระคุณ รองศาสตราจารย์ ดร.อนิราช มิ่งขวัญ ซึ่งเป็นอาจารย์ที่ปรึกษา ที่ได้กรุณาให้คำปรึกษา และแนะนำแนวทางในการจัดทำปริญญานิพนธ์ รวมทั้งคณาจารย์ภาควิชาเทคโนโลยีสารสนเทศที่ได้ให้ความรู้กับผู้จัดทำเพื่อนำมาประยุกต์ในการจัดทำปริญญานิพนธ์นี้
\\
\hspace*{1.5em}  %ย่อหน้า
ท้ายที่สุดนี้ ทางผู้จัดทำขอน้อมรำลึกพระคุณบิดา มารดา ผู้ซึ่งมีพระคุณอย่างสูงสุดที่ให้ความอุปการะผู้จัดทำมาโดยตลอด รวมทั้งผู้มีพระคุณทุกท่าน คณะผู้จัดทำรู้สึกซาบซึ้งเป็นอย่างยิ่ง จึงใคร่ขอขอบพระคุณเป็นอย่างสูงมา ณ โอกาสนี้ด้วย
\\
\begin{flushright}
\makebox[6cm][l]{ชัยสิทธิ์ มินทกร}      %ชื่อผู้จัดทำ1
%\\     %เว้นบรรทัด เมื่อมีเพื่อนร่วมโปรเจ็ค
%\makebox[6cm][l]{สิวาลัย จินเจือ}         %ชื่อผู้จัดทำ2

\end{flushright}