%*************************************
% บทที่ 1
%*************************************
\chapterTitle{1}{บทนำ}

%*******************************************
% ความเป็นมาและความสำคัญ
%*******************************************
\section{ความเป็นมาและความสำคัญ} 

%ระบุความเป็นมาและความสำคัญของโปรเจ็ค
\hspace*{1.5em} %ย่อหน้า
การบริหารจัดการและตั้งค่าระบบเครือข่าย (Network Configuration Management)
ในปัจจุบัน ส่วนใหญ่ยังคงอาศัยการสั่งงานผ่าน Command Line Interface (CLI) บนอุปกรณ์เครือข่าย Cisco ซึ่งมีความซับซ้อน ต้องอาศัยประสบการณ์สูง และมีความเสี่ยงต่อการเกิด Human Error โดยเฉพาะในกรณีที่ต้องปรับเปลี่ยนค่า Configuration จำนวนมาก หรือในสภาพแวดล้อมที่มีอุปกรณ์หลากหลาย
เทคโนโลยี Network Configuration Protocol (NETCONF) ได้ถูกพัฒนาขึ้นมาเป็นมาตรฐานใหม่สำหรับการจัดการอุปกรณ์เครือข่าย โดยใช้โปรโตคอลที่สามารถสื่อสารและจัดการข้อมูลได้อย่างมีระบบ ทำให้การปรับเปลี่ยนค่าคอนฟิกเป็นไปอย่าง อัตโนมัติ ตรวจสอบย้อนกลับได้ และมีประสิทธิภาพมากขึ้น อย่างไรก็ตาม การเขียนสคริปต์หรือโค้ดเพื่อควบคุม NETCONF ยังต้องอาศัยความรู้เฉพาะทาง
\hspace*{1.5em}
ด้วยเหตุนี้ การประยุกต์ใช้ Large Language Model (LLM) เข้ามาช่วยในการแปลงคำอธิบายหรือความต้องการของผู้ใช้งาน ให้กลายเป็นคำสั่ง NETCONF หรือ CLI แบบอัตโนมัติ จึงเป็นแนวทางใหม่ที่ช่วยให้ผู้ใช้งานทั่วไปสามารถบริหารจัดการและตั้งค่าอุปกรณ์ Cisco ได้อย่าง สะดวก รวดเร็ว และลดความผิดพลาด


%*******************************************
% วัตถุประสงค์ของโครงงาน
%*******************************************
\section{วัตถุประสงค์ของโครงงาน}

%ระบุวัตถุประสงค์ของโปรเจ็ค
\begin{mycustomenum}[label=1.2.\arabic*] %เพิ่มวัตถุประสงค์หลัง \item ถ้ามีหลายข้อ ก็เพิ่มบรรทัด \item ไปได้เรื่อยๆ
    \item เพื่อพัฒนาเว็บแอปพลิเคชันที่ช่วยให้ผู้ดูแลระบบสามารถจัดการและตั้งค่าอุปกรณ์ Cisco ผ่านโปรโตคอล NETCONF และ CLI ได้อย่างสะดวกและมีประสิทธิภาพ
    \item เพื่อประยุกต์ใช้เทคโนโลยี LLM ในการแปลงข้อความคำอธิบายของ ผู้ใช้ให้เป็นชุดคำสั่ง NETCONF หรือ CLI อัตโนมัติ 
    \item เพื่อลดความซับซ้อนและลดข้อผิดพลาดจากการตั้งค่าระบบเครือข่ายแบบ Manual ที่ต้องใช้ CLI ด้วยตนเอง
    \item เพื่อเพิ่มความรวดเร็ว ความถูกต้อง และความปลอดภัยในการบริหารจัดการอุปกรณ์เครือข่าย
    \item เพื่อศึกษาและประเมินศักยภาพของการนำ LLM มาประยุกต์ใช้ในงาน Network Automation 
\end{mycustomenum}



%*******************************************
% ขอบเขตของการทําโครงงานพิเศษ
%*******************************************
\section{ขอบเขตของการทําโครงงาน} \label{sec1:scope}
\hspace*{1 cm}\textbf{ภาคการศึกษาที่ 1/2568}
%ระบุขอบเขตของโปรเจ็ค
\begin{mycustomenum2}
    \item เว็บแอปพลิเคชันรองรับการเชื่อมต่อกับอุปกรณ์เครือข่ายใหม่ที่ยังไม่ได้ทำการตั้งค่าสำหรับ 
SSH โดยใช้การเชื่อมต่อ ผ่านทาง Console Port และสามารถทำการตั้งค่าต่าง ๆ โดยผู้ใช้สามารถกรอก
 ค่าเพื่่อให้ตรงกับความต้องการได้ 
    \begin{mycustomenum2}
        \item กำหนด Console Port ที่ต้องการเชื่อมต่อกับอุปกรณ์ (Router, Switch)
        \item กำหนด Baud rate ที่ตรงกับอุปกรณ์ 
        \item กำหนด Hostname ของอุปกรณ์ 
        \item กำหนด RSA key สำหรับ SSH (1024, 2048) 
        \item กำหนด IP domain-name 
        \item กำหนด Username, privilege และ password สำหรับ Login SSH 
        \item ระบุ Interface สำหรับ management ที่ต้องการตั้ง IP Address สำหรับเชื่อมต่อกับอุปกรณ์ในเครือข่ายสามารถเลือกได้ดังนี้
            \begin{mycustomenum2}
                \item ระบุ IP Address แบบ Manual
                \item แบบ DHCP 
            \end{mycustomenum2}
    \end{mycustomenum2}

    \item เว็บแอปพลิเคชันมีหน้า Dashboard สำหรับภาพรวมต่างๆของเว็บแอปพลิเคชัน 
    \begin{mycustomenum2}
        \item แสดงผลจำนวนอุปกรน์ทั้งหมด 
        \item แสดงผลข้อมูลอุปกรณ์แบบย่อ ได้แก่ Device Name, Type, IP Address, Location
        \item แสดงผลจำนวน Configuration History ทั้งหมด
        \item แสดงผลข้อมูล Configuration History แบบย่อ 5 Configuration ล่าสุด 
    \end{mycustomenum2}

    \item เว็บแอปพลิเคชันมีหน้า Devices Management สำหรับ เพิ่ม, แก้ไข, ลบ อุปกรณ์
    \begin{mycustomenum2}
        \item เพิ่มข้อมูลของอุปกรณ์สำหรับ SSH ได้แก่ IP address, Username, Password
        \item เพิ่มข้อมูลทั่วไปของอุปกรณ์ Type, Description, Location 
        \item จัดหมวดหมู่ของอุปกรณ์ Router, Switch 
        \item กำหนดสถานะของอุปกรณ์ Active, Inactive, Maintenance
    \end{mycustomenum2}

    \item เว็บแอปพลิเคชันมีหน้า LLM Configuration Generator สำหรับสร้าง Configuration ของอุปกรณ์ Cisco ด้วย LLM 
    \begin{mycustomenum2}
        \item สร้าง Template Configuration ให้กับ LLM
        \item สามารถเลือกอุปกรณ์ที่ต้องการจะ Config ได้  
        \vspace{4em}
        \item อุปกรณ์ที่สามารถ Config ได้แก่ Router, Switch L2, Switch L3 ของ Cisco 
            \begin{mycustomenum2}
                \item Basic Configuration ทั่วไปได้แก่ IP Address, Hostname, IP Route 
                \item Routing Protocol ได้แก่ OSPF, EIGRP, RIPv2, BGP, ISIS 
                \item VLAN Configuration บน Switch ได้แก่ VLAN Create, Trunk, Access 
            \end{mycustomenum2} 
        \item กำหนดสถานะของอุปกรณ์ Active, Inactive, Maintenance
    \end{mycustomenum2}

    \item เว็บแอปพลิเคชันมีหน้า Configuration History แสดงผลการ Generate Config ของ LLM ทั้งหมด 
    \begin{mycustomenum2}
        \item สามารถดูสถานะของ Configuration นั้น ๆ ได้แก่ Generated, Deployed, Failed
        \item สามารถดู Preview Configuration นั้น ๆ ได้ 
        \item สามารถลบ Configuration History ที่ไม่ต้องการได้
    \end{mycustomenum2}

    \item เว็บแอปพลิเคชันมีหน้า Backup Management สำหรับคืนค่า Configuration ของอุปกรณ์
    \begin{mycustomenum2}
        \item สามารถเพิ่ม Backup Configuration ได้
            \begin{mycustomenum2}
                \item เลือกอุปกรณ์ที่ต้องการจะ Backup Configuration (Router, Switch)
                \item กรอกข้อมูลทั่วไป Backup name, Description, Tags
                \item VLAN Configuration บน Switch ได้แก่ VLAN Create, Trunk, Access 
            \end{mycustomenum2} 
        \item มี Dashboard แสดงผล
            \begin{mycustomenum2}
                \item Total backup จำนวนของ Backup Configuration ทั้งหมด
                \item Backup type ประเภทของ Backup ได้แก่ Manual, Scheduled
                \item Total size จำนวนพื้นที่ของ Backup Configuration ใช้ทั้งหมด
            \end{mycustomenum2} 
        \item สามารถดูข้อมูลในแต่ละลิสต์ Backup Configuration ที่ Backup ไว้ 
    \end{mycustomenum2}

\end{mycustomenum2}


\hspace*{1 cm}\textbf{ภาคการศึกษาที่ 2/2568}
%ระบุขอบเขตของโปรเจ็ค
\begin{mycustomenum2}
    \item เว็บแอปพลิเคชันรองรับการเชื่อมต่อ NETCONF Client
    \begin{mycustomenum2}
        \item เว็บแอปพลิเคชันมีหน้า NETCONF/YANG Management
        \item พัฒนา Backend สำหรับเชื่อมต่ออุปกรณ์ผ่าน NETCONF over SSH (Port 830)
    \end{mycustomenum2}

    \item ฟีเจอร์ Active Sessions สำหรับจัดการ NETCONF Session 
    \begin{mycustomenum2}
        \item สร้างฟังก์ชันสำหรับจัดการ Session ของอุปกรณ์
            \begin{mycustomenum2}
                \item สามารถ Disconnect Session ของอุปกรณ์ได้
            \end{mycustomenum2}
        \item สามารถตรวจสอบ Capabilities ของอุปกรณ์ได้ 
    \end{mycustomenum2}

    \item ฟีเจอร์ NETCONF Device สำหรับจัดการอุปกรณ์ที่รองรับ NETCONF
    \begin{mycustomenum2}
        \item เพิ่มข้อมูลของอุปกรณ์สำหรับ NETCONF และ SSH ได้แก่ IP Address, Username, Password
        \item สร้างฟังก์ชันทดสอบการเชื่อมต่อกับอุปกรณ์ผ่าน NETCONF
    \end{mycustomenum2}

    \vspace{1em}
    \item ฟีเจอร์ YANG Models สำหรับจัดการ Model
    \begin{mycustomenum2}
        \item สามารเพิ่มหรือลบ YANG Models ได้ 
        \item สามารถอัปโหลดไฟล์ .yang สำหรับ YANG Models ที่เป็นมาตรฐานได้
        \item สามารถ Custom YANG Models ได้
        \item รองรับการค้นหา YANG Models ด้วยการ Search
        \item สามารถดูรายละเอียดของแต่ละ YANG Models ได้
    \end{mycustomenum2}

    \item ฟีเจอร์ Operations สำหรับดึงข้อมูลของอุปกรณ์ 
    \begin{mycustomenum2}
        \item สำหรับดึงข้อมูล Operational Data ได้แก่ Interfaces, Routing table, System info ผ่าน NETCONF <get> ของอุปกรณ์ Cisco Nexus 9000V และ CSR1000v 
        \item แสดงผลข้อมูลแบบ Real-time 
    \end{mycustomenum2}

    \item ฟีเจอร์ LLM XML Generator สำหรับ Generate XML
    \begin{mycustomenum2}
        \item ปรับปรุง LLM จากเดิม Generate CLI ให้ทำได้ทั้ง CLI และ XML
        \item สามารเลือกอุปกรณ์ที่ต้องการจะ Config ได้  
        \item พัฒนาฟังก์ชันใน LLM XML Generator สำหรับ Deploy XML
        \item พัฒนา LLM  สำหรับ Generate XML จากภาษาไทยและภาษาอังกฤษ
        \item สร้างฟังก์ชัน Validate XML กับ YANG Models ก่อน Deploy
    \end{mycustomenum2}

\end{mycustomenum2}



%*******************************************
% ประโยชน์ที่คาดว่าจะได้รับ
%*******************************************
\section{ประโยชน์ที่คาดว่าจะได้รับ}

%ระบุประโยชน์ที่คาดว่าจะได้รับของโปรเจ็ค
\begin{mycustomenum}[label=1.4.\arabic*] %เพิ่มวัตถุประสงค์หลัง \item ถ้ามีหลายข้อ ก็เพิ่มบรรทัด \item ไปได้เรื่อยๆ
    \item เพิ่มความสะดวกและรวดเร็วในการบริหารจัดการอุปกรณ์เครือข่าย ผ่าน Web Application ที่ออกแบบให้ใช้งานง่ายและรองรับผู้ใช้หลายระดับ 
    \item ลดความซับซ้อนในการเขียนคำสั่ง CLI/NETCONF โดยผู้ใช้ทั่วไปสามารถตั้งค่าระบบได้โดยไม่ต้องมีความรู้เชิงลึก 
    \item เพิ่มความถูกต้องและความปลอดภัยของการตั้งค่าระบบ ด้วยการใช้ LLM และระบบตรวจสอบคำสั่งอัตโนมัติ  
    \item สนับสนุนการเรียนรู้และพัฒนาทักษะของผู้จัดทำในด้าน Network Automation, LLM, Database และมาตรฐานสมัยใหม่
    \item ส่งเสริมการนำเทคโนโลยี LLM และมาตรฐาน NETCONF/YANG มาใช้จริง เพื่อเพิ่มประสิทธิภาพและความปลอดภัย  
    \item ช่วยให้การตัดสินใจและวางแผนการดูแลเครือข่ายในองค์กรแม่นยำมากขึ้น ด้วยข้อมูลที่ถูกต้องทันสมัย และตรวจสอบย้อนหลังได้ 
\end{mycustomenum}
