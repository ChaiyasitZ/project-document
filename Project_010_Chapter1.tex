%*************************************
% บทที่ 1
%*************************************
\chapterTitle{1}{บทนำ}

%*******************************************
% ความเป็นมาและความสำคัญ
%*******************************************
\section{ความเป็นมาและความสำคัญ} 

%ระบุความเป็นมาและความสำคัญของโปรเจ็ค
\hspace*{1.5em} %ย่อหน้า
ในยุคปัจจุบันที่การศึกษาและงานวิจัยมีความสำคัญอย่างยิ่งต่อการพัฒนาองค์ความรู้และความก้าวหน้าในสาขาต่าง ๆ การจัดทำเอกสารวิชาการที่มีคุณภาพและเป็นมาตรฐานจึงเป็นสิ่งจำเป็นอย่างยิ่ง โดยเฉพาะอย่างยิ่งในระดับอุดมศึกษาที่นักศึกษาต้องจัดทำปริญญานิพนธ์ เพื่อแสดงถึงความรู้ความเข้าใจและผลการศึกษาค้นคว้าของตนเอง อย่างไรก็ตาม การจัดทำรูปเล่มปริญญานิพนธ์มักเป็นกระบวนการที่ซับซ้อนและท้าทายสำหรับนักศึกษาจำนวนมาก ตั้งแต่การจัดการโครงสร้างเอกสาร การจัดรูปแบบให้เป็นไปตามระเบียบที่กำหนด การแทรกองค์ประกอบต่าง ๆ เช่น รูปภาพ ตาราง สมการ ไปจนถึงการจัดการบรรณานุกรมและการอ้างอิง ซึ่งหากดำเนินการด้วยโปรแกรมประมวลผลคำทั่วไป อาจก่อให้เกิดปัญหาด้านความไม่สอดคล้องกันของรูปแบบ ความผิดพลาดในการจัดเรียง หรือความยุ่งยากในการปรับแก้ไขเมื่อมีการเปลี่ยนแปลงเนื้อหา \parencite{kopka2004guide} ด้วยเหตุนี้ โปรแกรมประมวลผลเอกสาร LaTeX จึงได้เข้ามามีบทบาทสำคัญและเป็นที่ยอมรับอย่างกว้างขวางในแวดวงวิชาการและวิทยาศาสตร์ ด้วยความสามารถอันโดดเด่นในการจัดการเอกสารที่มีความซับซ้อนได้อย่างมีประสิทธิภาพ ความแม่นยำในการจัดรูปแบบที่เป็นมาตรฐาน และความสามารถในการสร้างสรรค์เอกสารที่มีความเป็นมืออาชีพสูง ทำให้ LaTeX เป็นเครื่องมือที่ตอบโจทย์ความต้องการในการจัดทำเอกสารวิชาการได้อย่างสมบูรณ์แบบ \parencite{lamport1994latex, latexprojectabout}

\hspace*{1.5em}
ดังนั้น โครงงาน "การจัดทำรูปเล่มปริญญานิพนธ์ด้วย LaTeX" จึงได้ริเริ่มขึ้น โดยมีวัตถุประสงค์เพื่อนำเสนอ แนวทางการปฏิบัติที่เป็นรูปธรรมและเข้าใจง่าย สำหรับนักศึกษา เพื่อช่วยให้นักศึกษาสามารถใช้ประโยชน์จาก LaTeX ในการจัดทำรูปเล่มปริญญานิพนธ์ได้อย่างมีประสิทธิภาพ ตั้งแต่โครงสร้างพื้นฐานของเอกสารไปจนถึงรายละเอียดปลีกย่อยต่าง ๆ ที่จำเป็น เพื่อให้ได้มาซึ่งรูปเล่มปริญญานิพนธ์ที่มีคุณภาพ ถูกต้องตามหลักวิชาการ และเป็นไปตามมาตรฐานที่กำหนด \parencite{gratzer2008practical}



%*******************************************
% วัตถุประสงค์ของโครงงาน
%*******************************************
\section{วัตถุประสงค์ของโครงงาน}

%ระบุวัตถุประสงค์ของโปรเจ็ค
\begin{mycustomenum}[label=1.2.\arabic*] %เพิ่มวัตถุประสงค์หลัง \item ถ้ามีหลายข้อ ก็เพิ่มบรรทัด \item ไปได้เรื่อยๆ
    \item เพื่อนำเสนอแนวทางปฏิบัติในการจัดทำเอกสารรูปเล่มปริญญานิพนธ์ด้วย LaTeX
    \item เพื่ออำนวยความสะดวกแก่นักศึกษาในการจัดทำรูปเล่มปริญญานิพนธ์
    \item เพื่อเป็นคู่มือที่ใช้ประกอบการเรียนรู้และอ้างอิง
\end{mycustomenum}



%*******************************************
% ขอบเขตของการทําโครงงานพิเศษ
%*******************************************
\section{ขอบเขตของการทําโครงงาน} \label{sec1:scope}
\hspace*{1 cm}\textbf{ภาคการศึกษาที่ 1/2568}
%ระบุขอบเขตของโปรเจ็ค
\begin{mycustomenum2}
    \item ด้านเนื้อหา
    \begin{mycustomenum2}
        \item ครอบคลุมองค์ประกอบหลักของการจัดทำรูปเล่มปริญญานิพนธ์ฉบับสมบูรณ์ ตั้งแต่หน้าปกไปจนถึงภาคผนวก ได้แก่
        \begin{mycustomenum2}
            \item บทคัดย่อ (ภาษาไทยและภาษาอังกฤษ)
            \item กิตติกรรมประกาศ
            \item สารบัญ สารบัญภาพ สารบัญตาราง
            \item เนื้อหาหลัก ได้แก่ 
            \begin{mycustomenum2}
                \item บทนำ
                \item แนวคิด ทฤษฎี และงานวิจัยที่เกี่ยวข้อง
                \item วิธีการดำเนินงาน
                \item ผลการดำเนินงาน
                \item สรุปผลการดำเนินงาน
            \end{mycustomenum2}            
            \item บรรณานุกรม
            \item ภาคผนวก
        \end{mycustomenum2}
        \item เน้นการใช้งานคำสั่งและแพ็กเกจที่จำเป็นใน LaTeX สำหรับการจัดรูปแบบเอกสารวิชาการ เช่น การจัดการหัวเรื่อง การแทรกรูปภาพ การสร้างตาราง การเขียนสมการ และการจัดการระบบการอ้างอิงและบรรณานุกรม (โดยเฉพาะสไตล์ APA)
        \item นำเสนอตัวอย่างโค้ด LaTeX ที่สามารถนำไปประยุกต์ใช้ได้จริงในแต่ละส่วนของปริญญานิพนธ์
    \end{mycustomenum2}

    \item ด้านเครื่องมือและแพลตฟอร์ม
    \begin{mycustomenum2}
        \item มุ่งเน้นการใช้งานโปรแกรมประมวลผลเอกสาร LaTeX เป็นหลัก
        \item แนะนำการใช้งาน TeX Distribution ที่เป็นที่นิยม เช่น MiKTeX หรือ TeX Live
        \item แนะนำการใช้ Integrated Development Environment (IDE) สำหรับ LaTeX เช่น Overleaf เพื่อความสะดวกในการทำงาน
    \end{mycustomenum2}
\end{mycustomenum2}


\hspace*{1 cm}\textbf{ภาคการศึกษาที่ 2/2568}
%ระบุขอบเขตของโปรเจ็ค
\begin{mycustomenum2}
    \item ด้านกลุ่มเป้าหมาย
    \begin{mycustomenum2}
        \item นักศึกษาในระดับปริญญาตรีที่กำลังจัดทำปริญญานิพนธ์ รายงานวิจัย หรือเอกสารวิชาการอื่น ๆ
        \item ผู้ที่สนใจเรียนรู้การใช้งาน LaTeX เพื่อการจัดทำเอกสารอย่างเป็นระบบและมีมาตรฐาน
    \end{mycustomenum2}

    \item ด้านสิ่งที่ไม่ครอบคลุม
    \begin{mycustomenum2}
        \item โครงงานนี้จะไม่เน้นเนื้อหาเชิงลึกเกี่ยวกับการเขียนโค้ด LaTeX สำหรับการพัฒนาแพ็กเกจใหม่ ๆ หรือการปรับแต่งสไตล์ที่ไม่ใช่มาตรฐาน
        \item จะไม่รวมเนื้อหาเกี่ยวกับการวิเคราะห์ข้อมูล หรือการเขียนเนื้อหางานวิจัยเชิงวิชาการโดยตรง แต่จะเน้นที่การจัดรูปแบบของเอกสารนั้น ๆ
    \end{mycustomenum2}
\end{mycustomenum2}



%*******************************************
% ประโยชน์ที่คาดว่าจะได้รับ
%*******************************************
\section{ประโยชน์ที่คาดว่าจะได้รับ}

%ระบุประโยชน์ที่คาดว่าจะได้รับของโปรเจ็ค
\begin{mycustomenum}[label=1.4.\arabic*] %เพิ่มวัตถุประสงค์หลัง \item ถ้ามีหลายข้อ ก็เพิ่มบรรทัด \item ไปได้เรื่อยๆ
    \item ได้นำเสนอแนวทางปฏิบัติในการจัดทำเอกสารรูปเล่มปริญญานิพนธ์ด้วย LaTeX
    \item ได้อำนวยความสะดวกแก่นักศึกษาในการจัดทำรูปเล่มปริญญานิพนธ์
    \item ได้คู่มือที่ใช้ประกอบการเรียนรู้และอ้างอิง
\end{mycustomenum}
