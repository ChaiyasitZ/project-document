%*************************************
% บทที่ 5
%*************************************
\chapterTitle{5}{สรุปผลการดำเนินงาน}

\section{สรุปผลการดำเนินงาน}
\hspace*{1.5em}
โครงการ "การจัดทำรูปเล่มปริญญานิพนธ์ด้วย LaTeX" ได้พัฒนาแม่แบบ (Template) สำหรับการเขียนรูปเล่มปริญญานิพนธ์โดยใช้ LaTeX โดยมีการรวบรวมคำสั่งพื้นฐานที่จำเป็นและปรับโครงสร้างเอกสารให้สอดคล้องกับมาตรฐานของภาควิชาเทคโนโลยีสารสนเทศ ผลลัพธ์ที่สำคัญของโครงการคือการสร้างแม่แบบ LaTeX ที่สมบูรณ์ ซึ่งครอบคลุมทุกส่วนของรูปเล่มปริญญานิพนธ์ ตั้งแต่หน้าปก สารบัญ สารบัญรูปภาพ สารบัญตาราง เนื้อหา ภาคผนวก และบรรณานุกรม ทำให้ผู้ใช้สามารถนำไปใช้งานได้ทันทีและช่วยลดเวลาในการจัดรูปแบบเอกสารได้อย่างมาก ส่งผลให้ผู้จัดทำมุ่งเน้นไปที่การเขียนเนื้อหาทางวิชาการได้อย่างเต็มที่

\section{ปัญหาและอุปสรรค}
\begin{mycustomenum}[label=5.2.\arabic*] 
    \item ความซับซ้อนของ LaTeX เนื่องจาก LaTeX มีคำสั่งและโครงสร้างที่ต้องเรียนรู้มากมาย ทำให้ผู้เริ่มต้นใช้งานต้องใช้เวลาในการทำความเข้าใจ
    \item การจัดบรรณานุกรม (Bibliography) ไฟล์ .bib ถูกใช้เพื่อจัดการข้อมูลบรรณานุกรม อาจมีความซับซ้อนสำหรับผู้ใช้ที่ยังไม่คุ้นเคย
    \item การจัดการรูปภาพและตาราง ในการปรับขนาดและตำแหน่งของรูปภาพหรือตารางให้เหมาะสมในเอกสารเป็นเรื่องที่ต้องอาศัยการลองผิดลองถูกพอสมควร
    \item การแก้ไขข้อผิดพลาด (Errors) เมื่อเกิดข้อผิดพลาดในเอกสาร LaTeX การทำความเข้าใจและแก้ไขข้อผิดพลาดเหล่านั้นอาจเป็นเรื่องที่ท้าทาย
\end{mycustomenum}

\section{การแก้ปัญหา}
\begin{mycustomenum}[label=5.3.\arabic*] 
    \item จัดทำตัวอย่างของการนำ LaTex มาใช้งานพร้อมตัวอย่างโค้ดที่ง่ายต่อการทำความเข้าใจ พร้อมภาพประกอบและคำอธิบายทีละขั้นตอน
    \item เลือกใช้แพ็กเกจที่เหมาะสม เลือกใช้แพ็กเกจ LaTeX ที่ใช้งานง่ายและเป็นที่นิยม 
    \item สร้างโค้ดตัวอย่างสำหรับรูปแบบการจัดวางรูปภาพและตารางที่ใช้บ่อย เช่น โค้ดสำหรับรูปภาพขนาดเต็มหน้า โค้ดสำหรับตารางที่มีขนาดพอดีกับหน้ากระดาษ เพื่อให้ผู้ใช้สามารถคัดลอกและนำไปปรับใช้ได้ทันที
\end{mycustomenum}

\section{ข้อเสนอแนะ}
\begin{mycustomenum}[label=5.4.\arabic*] 
    \item พัฒนาฟังก์ชันการใช้งานเพิ่มเติม เช่น การสร้างดัชนี (Index) หรือการเพิ่มความสามารถในการปรับแต่งรูปแบบต่างๆ ให้หลากหลายมากขึ้น
    \item จัดอบรมเชิงปฏิบัติการเพื่อสอนการใช้งาน LaTeX เพื่อให้สามารถใช้งานโปรแกรมได้อย่างมีประสิทธิภาพมากขึ้น 
\end{mycustomenum}