%*************************************
% บทที่ 2
%*************************************
\chapterTitle{2}{แนวคิด ทฤษฎี และงานวิจัยที่เกี่ยวข้อง}

%*******************************************
% แนวคิด ทฤษฎีต่างๆ
%*******************************************
\section{แนวคิดพื้นฐานเกี่ยวกับโครงงาน}

% 2.1.1
\subsection{ความสำคัญของปริญญานิพนธ์และเอกสารวิชาการ}

\hspace*{1.5em} %ย่อหน้า
การศึกษาในระดับอุดมศึกษา ไม่ว่าจะเป็นระดับปริญญาตรี หรือสูงกว่านั้น ล้วนมีจุดมุ่งหมายสำคัญในการพัฒนาศักยภาพของผู้เรียนให้สามารถสร้างสรรค์องค์ความรู้ใหม่ หรือต่อยอดความรู้เดิมให้เกิดประโยชน์ต่อสังคมและวิชาการ โดยมีปริญญานิพนธ์หรือที่อาจเรียกว่าวิทยานิพนธ์ สารนิพนธ์ หรือโครงงานพิเศษ เป็นหัวใจสำคัญที่สะท้อนถึงการบรรลุเป้าหมายดังกล่าว

\hspace*{1.5em} %ย่อหน้า
ปริญญานิพนธ์เป็นมากกว่ารายงานฉบับหนึ่ง เพราะมันคือ ผลผลิตทางปัญญา ที่แสดงออกถึงความสามารถของผู้ศึกษาในการ
\begin{mycustomitem}
    \item \textbf{สังเคราะห์องค์ความรู้} ด้วยการรวบรวม วิเคราะห์ และสังเคราะห์ข้อมูลจากแหล่งต่าง ๆ เพื่อให้ได้มาซึ่งความเข้าใจที่ลึกซึ้งในสาขาวิชานั้น ๆ
    \item \textbf{สร้างองค์ความรู้ใหม่} ด้วยการทำวิจัย ทดลอง หรือพัฒนาสิ่งประดิษฐ์ เพื่อสร้างองค์ความรู้ที่ไม่เคยมีมาก่อน หรือพิสูจน์สมมติฐานใหม่
    \item \textbf{แก้ปัญหาทางวิชาการหรือสังคม} ด้วยการนำความรู้ที่ได้มาประยุกต์ใช้เพื่อเสนอแนวทางแก้ไขปัญหาที่เกิดขึ้นจริง
    \item \textbf{สื่อสารผลการศึกษาอย่างมีประสิทธิภาพ} ด้วยการจัดระเบียบความคิด และนำเสนอผลการศึกษาค้นคว้าอย่างเป็นระบบ ถูกต้องตามหลักวิชาการ และเข้าใจง่าย
\end{mycustomitem}

\hspace*{1.5em} %ย่อหน้า
อย่างไรก็ตาม การจัดทำปริญญานิพนธ์และเอกสารวิชาการที่มีคุณภาพและเป็นมาตรฐานนั้นไม่ใช่เรื่องง่าย นักศึกษาจำนวนมากต้องเผชิญกับความท้าทายหลายประการ ได้แก่
\begin{mycustomitem}
    \item \textbf{ความซับซ้อนของโครงสร้าง} เอกสารวิชาการโดยเฉพาะปริญญานิพนธ์ มีโครงสร้างและส่วนประกอบที่ตายตัว ตั้งแต่บทคัดย่อ สารบัญ เนื้อหาหลัก บรรณานุกรม ไปจนถึงภาคผนวก ซึ่งต้องจัดเรียงให้ถูกต้องและเป็นระเบียบ
    \item \textbf{ความแม่นยำในการจัดรูปแบบ} การกำหนดรูปแบบตัวอักษร การเว้นวรรค ระยะห่าง การจัดหน้า การใส่หัว-ท้ายกระดาษ และการใช้สไตล์ (เช่น APA, MLA, Chicago) ล้วนต้องมีความละเอียดและสม่ำเสมอ ซึ่งเป็นข้อกำหนดที่เข้มงวดในการเผยแพร่ผลงานวิชาการ
    \item \textbf{การจัดการองค์ประกอบที่หลากหลาย} การแทรกรูปภาพ ตาราง กราฟ หรือสมการทางคณิตศาสตร์ให้สวยงาม อ่านง่าย และสามารถอ้างอิงถึงได้ในเนื้อหา เป็นอีกหนึ่งความท้าทายที่ต้องอาศัยความเข้าใจในเครื่องมือ
    \item \textbf{การจัดการบรรณานุกรมและการอ้างอิง} การอ้างอิงแหล่งที่มาของข้อมูลอย่างถูกต้องและเป็นระบบเป็นสิ่งสำคัญยิ่งในการแสดงความน่าเชื่อถือทางวิชาการและป้องกันการคัดลอกผลงาน ซึ่งมักจะต้องปฏิบัติตามรูปแบบที่กำหนดอย่างเคร่งครัด
    \item \textbf{การบริหารจัดการเอกสารขนาดใหญ่} ปริญญานิพนธ์ส่วนใหญ่มีความยาวหลายสิบหรือหลายร้อยหน้า การแก้ไขเนื้อหาเพียงเล็กน้อยในจุดหนึ่งอาจส่งผลกระทบต่อการจัดหน้าทั้งหมดของเอกสาร ทำให้การจัดการเอกสารเป็นไปอย่างยากลำบากหากใช้เครื่องมือที่ไม่เหมาะสม
\end{mycustomitem}



% 2.1.2
\subsection{บทบาทของเทคโนโลยีในการจัดทำเอกสารวิชาการ}

\hspace*{1.5em} %ย่อหน้า
ในอดีต การจัดทำเอกสารวิชาการต้องอาศัยการเขียนด้วยลายมือหรือการพิมพ์ดีด ซึ่งมีข้อจำกัดอย่างมากในด้านความเร็ว ความยืดหยุ่น และความแม่นยำในการจัดรูปแบบ อย่างไรก็ตาม การพัฒนาของเทคโนโลยีสารสนเทศได้ปฏิวัติวิธีการจัดทำเอกสาร โดยเฉพาะอย่างยิ่งการเข้ามาของโปรแกรมประมวลผลคำและระบบจัดเตรียมเอกสารต่าง ๆ

\hspace*{1.5em} %ย่อหน้า
เทคโนโลยีได้เข้ามามีบทบาทสำคัญในการช่วยลดภาระงานด้านการจัดรูปแบบและเพิ่มประสิทธิภาพในการผลิตเอกสารวิชาการหลายประการ ได้แก่
\begin{mycustomitem}
    \item \textbf{ลดเวลาและข้อผิดพลาดในการจัดรูปแบบ} โปรแกรมคอมพิวเตอร์สามารถดำเนินการจัดรูปแบบได้อย่างรวดเร็วและสม่ำเสมอ ลดโอกาสเกิดข้อผิดพลาดที่เกิดจากการจัดทำด้วยมือ
    \item \textbf{เพิ่มความยืดหยุ่นในการแก้ไข} การแก้ไขเนื้อหาหรือโครงสร้างเอกสารทำได้ง่ายและรวดเร็ว โดยไม่จำเป็นต้องพิมพ์ใหม่ทั้งหมด
    \item \textbf{รองรับองค์ประกอบที่ซับซ้อน} เทคโนโลยีช่วยให้สามารถแทรกและจัดการรูปภาพ ตาราง กราฟ และสมการคณิตศาสตร์ได้อย่างมีประสิทธิภาพ ซึ่งเป็นสิ่งที่ยากจะทำได้ด้วยวิธีการแบบดั้งเดิม
    \item \textbf{สร้างความสอดคล้องและมาตรฐาน} โปรแกรมประมวลผลเอกสารสามารถใช้เทมเพลตหรือสไตล์ชีทเพื่อบังคับใช้รูปแบบที่เป็นมาตรฐาน ทำให้เอกสารมีความสอดคล้องกันทั้งเล่ม
    \item \textbf{สะดวกในการอ้างอิง} ระบบจัดการบรรณานุกรมในซอฟต์แวร์ ช่วยให้การอ้างอิงเป็นไปโดยอัตโนมัติ และช่วยลดความผิดพลาดและประหยัดเวลาอย่างมาก
\end{mycustomitem}

\hspace*{1.5em} %ย่อหน้า
แม้ว่าจะมีโปรแกรมประมวลผลคำหลายประเภทที่ใช้กันอย่างแพร่หลาย แต่ ระบบจัดเตรียมเอกสารเฉพาะทาง อย่าง LaTeX ได้รับการยอมรับในแวดวงวิชาการเป็นพิเศษ เนื่องจากมีจุดเด่นในการจัดการเอกสารที่มีความซับซ้อนสูง และให้ผลลัพธ์ที่มีคุณภาพการพิมพ์ระดับมืออาชีพ ซึ่งตอบโจทย์ความต้องการในการจัดทำปริญญานิพนธ์และเอกสารวิชาการได้อย่างแท้จริง



%2.2
\section{แนวคิดและทฤษฎีที่เกี่ยวข้องกับ LaTeX}
%2.2.1
\subsection{ความเป็นมาและวิวัฒนาการของ LaTeX}

\hspace*{1.5em} %ย่อหน้า
การเริ่มต้นของ LaTeX ไม่สามารถแยกออกจาก TeX ได้ เนื่องจาก LaTeX เป็นระบบที่สร้างขึ้นบนพื้นฐานของ TeX เพื่อให้การใช้งาน TeX ง่ายขึ้นและมีประสิทธิภาพยิ่งขึ้น

\hspace*{1.5em} %ย่อหน้า
\textbf{TeX (เท็กซ์)} %สร้างหัวข้อเป็นตัวหนา
\begin{mycustomitem}
    \item \textbf{กำเนิดโดย Donald E. Knuth} TeX ถูกสร้างขึ้นโดยศาสตราจารย์ Donald E. Knuth (โดนัลด์ อี. คนูท) แห่งมหาวิทยาลัยสแตนฟอร์ด ซึ่งเป็นนักวิทยาศาสตร์คอมพิวเตอร์ผู้มีชื่อเสียงระดับโลกในช่วงปลายทศวรรษ 1970
    \item \textbf{แรงจูงใจในการสร้าง} Knuth ไม่พอใจกับคุณภาพของการเรียงพิมพ์ (typesetting) ในหนังสือและบทความทางคณิตศาสตร์ในยุคนั้น โดยเฉพาะอย่างยิ่งการจัดรูปแบบสมการที่ซับซ้อน เขาจึงตัดสินใจสร้างระบบการเรียงพิมพ์ของตัวเองที่สามารถผลิตเอกสารที่มีคุณภาพสูงระดับสิ่งพิมพ์ได้ (publication-quality)
    \item \textbf{เป้าหมาย} TeX ถูกออกแบบมาเพื่อสร้างเอกสารทางวิทยาศาสตร์และวิศวกรรมที่มีความถูกต้องแม่นยำทางคณิตศาสตร์และสวยงามทางศิลปะการเรียงพิมพ์ Knuth ได้อุทิศเวลาหลายปีให้กับโครงการนี้ และได้เผยแพร่ TeX เวอร์ชันแรกในปี 1978 และเวอร์ชันที่สมบูรณ์มากขึ้นคือ TeX82 ในปี 1982
    \item \textbf{ลักษณะเฉพาะของ TeX} TeX เป็นโปรแกรม "เรียงพิมพ์" (typesetting program) ที่ทำงานในระดับต่ำ (low-level) คือผู้ใช้ต้องป้อนคำสั่ง (macros) เพื่อควบคุมการจัดรูปแบบทุกอย่างด้วยตนเอง ซึ่งมีความซับซ้อนและต้องใช้ทักษะการเขียนโปรแกรมในระดับหนึ่ง แม้จะทรงพลังมาก แต่ก็เป็นอุปสรรคต่อผู้ใช้ทั่วไป
\end{mycustomitem}

\hspace*{1.5em} %ย่อหน้า
\textbf{LaTeX (ลาเท็กซ์)} %สร้างหัวข้อเป็นตัวหนา
\begin{mycustomitem}
    \item \textbf{กำเนิดโดย Leslie Lamport} ในช่วงกลางทศวรรษ 1980 Leslie Lamport (เลสลี่ แลมพอร์ต) นักวิทยาศาสตร์คอมพิวเตอร์ชาวอเมริกัน ได้พัฒนาชุดคำสั่ง macros สำหรับ TeX ขึ้นมา เพื่อลดความซับซ้อนในการใช้งาน TeX และทำให้ผู้ใช้สามารถจัดทำเอกสารโดยเน้นที่โครงสร้างเนื้อหามากกว่าการจัดรูปแบบปลีกย่อย ระบบนี้ถูกเรียกว่า LaTeX ซึ่งย่อมาจาก "Lamport TeX"
    \item \textbf{ปรัชญาของ LaTeX} LaTeX เป็นเหมือน "ภาษาทำเครื่องหมาย" (markup language) คล้ายกับ HTML ที่ผู้ใช้เขียนคำสั่งแทรกไปในเนื้อหาเพื่อกำหนดโครงสร้าง เช่น บท ตอน หัวข้อ รายการ ตาราง และสมการ โดยไม่ต้องกังวลกับรายละเอียดการจัดหน้า คำสั่งเหล่านี้จะถูกประมวลผลโดย TeX engine เพื่อสร้างผลลัพธ์ที่สวยงามและเป็นระเบียบ
    \item \textbf{วิวัฒนาการของ LaTeX} LaTeX ได้รับการพัฒนาอย่างต่อเนื่อง และมีการอัปเดตเวอร์ชันเรื่อยมา ปัจจุบันเวอร์ชันที่ใช้กันอย่างแพร่หลายคือ LaTeX2e ซึ่งมีการปรับปรุงและเพิ่มความสามารถต่างๆ มากมาย เพื่อรองรับการใช้งานที่หลากหลายและซับซ้อนยิ่งขึ้น มีการพัฒนา package เพิ่มเติมจำนวนมากโดยชุมชนผู้ใช้งานทั่วโลก เพื่อขยายขีดความสามารถของ LaTeX ในด้านต่างๆ เช่น การวาดภาพ การสร้างกราฟ การจัดการบรรณานุกรม การรองรับภาษาต่างๆ (เช่น XeLaTeX สำหรับภาษาไทย) และอื่นๆ
\end{mycustomitem}


%2.2.2
\subsection{ความนิยมและการยอมรับในแวดวงวิชาการและวิทยาศาสตร์}
\hspace*{1.5em} %ย่อหน้า
LaTeX ได้รับการยอมรับอย่างกว้างขวางและกลายเป็นมาตรฐาน \textit{de facto} ในการจัดทำเอกสารทางวิชาการและวิทยาศาสตร์ด้วยเหตุผลหลายประการ

%\textit คือการทำตัวเอียง
\hspace*{1.5em} %ย่อหน้า
\textbf{คุณภาพของการจัดเรียงเนื้อหา}
    \begin{mycustomitem}
        \item \textbf{สมการคณิตศาสตร์} LaTeX มีความสามารถโดดเด่นในการจัดรูปแบบสมการคณิตศาสตร์ที่ซับซ้อนให้สวยงามและอ่านง่าย ซึ่งเป็นสิ่งสำคัญอย่างยิ่งในสาขาวิทยาศาสตร์ วิศวกรรม และคณิตศาสตร์ การจัดเรียงสัญลักษณ์ ตัวยก ตัวห้อย เศษส่วน เมทริกซ์ และโครงสร้างทางคณิตศาสตร์อื่นๆ ทำได้อย่างไร้ที่ติ
        \item \textbf{การจัดหน้าอัตโนมัติ} LaTeX สามารถจัดการการจัดหน้า การแบ่งย่อหน้า การสร้างสารบัญ ดัชนี การอ้างอิงไขว้ (cross-referencing) และบรรณานุกรมได้อย่างอัตโนมัติ ทำให้ผู้ใช้ไม่ต้องเสียเวลาปรับแต่งด้วยตนเอง และมั่นใจได้ว่าเอกสารจะเป็นระเบียบสอดคล้องกันทั้งเล่ม
        \item \textbf{คุณภาพของตัวอักษรและรูปภาพ} TeX/LaTeX ใช้แนวคิดของ Metafont ในการสร้างฟอนต์ และรองรับการแทรกรูปภาพคุณภาพสูง ทำให้ได้ผลลัพธ์ที่เป็นมืออาชีพ
    \end{mycustomitem}
    
\hspace*{1.5em} %ย่อหน้า
\textbf{ความคงเส้นคงวาและแม่นยำ}
    \begin{mycustomitem}
        \item \textbf{ความเป็นมาตรฐาน} เมื่อกำหนดรูปแบบเอกสารด้วย LaTeX แล้ว ผลลัพธ์ที่ได้จะมีความคงเส้นคงวาไม่ว่าจะเปิดในระบบปฏิบัติการใด หรือใช้เวอร์ชันใดของ LaTeX (ตราบใดที่ใช้แพ็กเกจและคอมไพเลอร์ที่เข้ากันได้) ซึ่งต่างจากโปรแกรมประมวลผลคำทั่วไปที่อาจเกิดปัญหาการจัดรูปแบบเพี้ยนเมื่อเปิดต่างเวอร์ชันหรือต่างแพลตฟอร์ม
        \item \textbf{ควบคุมได้ละเอียด} แม้จะเน้นโครงสร้าง แต่ LaTeX ก็ยังให้ความสามารถในการควบคุมรายละเอียดการจัดรูปแบบได้สูงมาก หากผู้ใช้ต้องการปรับแต่งเฉพาะจุด
    \end{mycustomitem}

\hspace*{1.5em} %ย่อหน้า
\textbf{ความสามารถในการทำงานร่วมกัน}
    \begin{mycustomitem}
        \item \textbf{ไฟล์ข้อความธรรมดา} เอกสาร LaTeX เป็นไฟล์ข้อความธรรมดา (.tex) ทำให้ง่ายต่อการจัดการด้วยระบบควบคุมเวอร์ชัน (Version Control Systems) เช่น Git ซึ่งเป็นที่นิยมในหมู่นักพัฒนาซอฟต์แวร์และนักวิจัย ทำให้หลายคนสามารถทำงานบนเอกสารเดียวกันได้โดยไม่เกิดความขัดแย้ง
        \item \textbf{แบ่งส่วนเอกสาร} การแยกเอกสารออกเป็นไฟล์ย่อยๆ (เช่น แยกแต่ละบทเป็นไฟล์ .tex) ทำให้การทำงานเป็นทีมสะดวกยิ่งขึ้น
    \end{mycustomitem}

\hspace*{1.5em} %ย่อหน้า
\textbf{เป็น Open Source และฟรี}
    \begin{mycustomitem}
        \item \textbf{ใช้งานได้ฟรี} LaTeX เป็นซอฟต์แวร์โอเพนซอร์สที่ใช้งานได้ฟรี ทำให้เข้าถึงได้ทุกคน ไม่ว่าจะเป็นนักศึกษา นักวิจัย หรือสถาบันการศึกษา ซึ่งลดภาระค่าใช้จ่ายด้านซอฟต์แวร์
        \item \textbf{ชุมชนผู้ใช้งานขนาดใหญ่} มีชุมชนผู้ใช้งานทั่วโลกที่คอยให้การสนับสนุน พัฒนาแพ็กเกจใหม่ๆ และตอบคำถาม ทำให้เกิดการเรียนรู้และแก้ปัญหาร่วมกันได้ง่าย
    \end{mycustomitem}

\hspace*{1.5em} %ย่อหน้า
\textbf{ความเหมาะสมสำหรับเอกสารวิชาการ}
    \begin{mycustomitem}
        \item \textbf{การอ้างอิงและบรรณานุกรม} LaTeX มีเครื่องมือจัดการการอ้างอิงและบรรณานุกรมที่มีประสิทธิภาพสูง (เช่น BibTeX, BibLaTeX) ทำให้การจัดการแหล่งอ้างอิงจำนวนมากเป็นเรื่องง่ายและแม่นยำตามรูปแบบมาตรฐานต่าง ๆ
        \item \textbf{โครงสร้างที่ซับซ้อน} เหมาะสำหรับการสร้างเอกสารที่มีโครงสร้างซับซ้อน เช่น วิทยานิพนธ์ บทความวิจัย หนังสือ หรือรายงานทางเทคนิค ที่มีส่วนประกอบจำนวนมาก เช่น บท ภาคผนวก สารบัญรูปภาพ สารบัญตาราง
        \item \textbf{มาตรฐานสิ่งพิมพ์} สำนักพิมพ์ทางวิชาการและวารสารวิทยาศาสตร์จำนวนมากยอมรับและบางแห่งกำหนดให้ส่งต้นฉบับในรูปแบบ LaTeX เพราะสามารถนำไปจัดพิมพ์ต่อได้ง่ายและได้คุณภาพสูง
    \end{mycustomitem}

\hspace*{1.5em} %ย่อหน้า
แม้ว่าการเรียนรู้ LaTeX ในช่วงเริ่มต้นผู้ใช้งานอาจต้องใช้เวลาในการเรียนรู้มากกว่าโปรแกรมประมวลผลคำทั่วไป แต่ด้วยข้อดีที่กล่าวมาทั้งหมดนี้ ทำให้ LaTeX ยังคงเป็นเครื่องมือที่สำคัญและขาดไม่ได้ LaTeX ได้รับความนิยมอย่างสูงในแวดวงวิชาการและวิทยาศาสตร์ทั่วโลกจนถึงปัจจุบัน โดยเฉพาะอย่างยิ่งในสาขาที่ต้องมีการนำเสนอสมการคณิตศาสตร์และโครงสร้างที่ซับซ้อนได้อย่างแม่นยำและสวยงาม

%2.2.3
\subsection{สถาปัตยกรรมและหลักการทำงานของ LaTeX}

\hspace*{1.5em} %ย่อหน้า
ในโลกของการจัดพิมพ์เอกสารทางวิทยาศาสตร์และเทคนิค LaTeX ได้รับการยอมรับอย่างกว้างขวางว่าเป็นเครื่องมือที่ทรงพลังและแม่นยำ แต่น้อยคนนักที่จะรู้ถึง สถาปัตยกรรม และ หลักการทำงานเบื้องหลัง ที่ทำให้มันแตกต่างและมีประสิทธิภาพเหนือกว่าโปรแกรมประมวลผลคำทั่วไป

\hspace*{1.5em} %ย่อหน้า
\textbf{การทำงานแบบ Markup Language (คำสั่งและสภาพแวดล้อม)}

\hspace*{1.5em} %ย่อหน้า
LaTeX ไม่ใช่โปรแกรมประมวลผลคำแบบเห็นผลลัพธ์ทันที แต่ทำงานในลักษณะของ \textbf{Markup Language} คล้ายกับ HTML หรือ Markdown หลักการคือจะเขียนข้อความและแทรก \textbf{คำสั่ง (Commands)} หรือกำหนด \textbf{สภาพแวดล้อม (Environments)} เพื่อบอกให้ LaTeX รู้ว่าส่วนนั้น ๆ ของเอกสารควรมีลักษณะอย่างไร ไม่ใช่การจัดรูปแบบด้วยมือโดยตรง

\begin{mycustomitem}
    \item \textbf{คำสั่ง (Commands)} คำสั่งใน LaTeX มักเริ่มต้นด้วยเครื่องหมายแบ็กสแลช (\textbackslash) ตามด้วยชื่อคำสั่ง เช่น
    \begin{mycustomitem}
        \item \textbf{\textbackslash section\{ชื่อหัวข้อ\}} ใช้สำหรับกำหนดหัวข้อหลักของส่วนนั้น ๆ
        \item \textbf{\textbackslash textbf\{ตัวหนา\}} ใช้สำหรับทำให้ข้อความเป็นตัวหนา
        \item \textbf{\textbackslash textbackslash} ใช้สำหรับแสดงเครื่องหมาย \textbackslash
        \item \textbf{\textbackslash includegraphics\{รูปภาพ.png\}} ใช้สำหรับแทรกรูปภาพ
    \end{mycustomitem}

คำสั่งเหล่านี้จะถูกประมวลผลเมื่อคอมไพล์เอกสาร เพื่อสร้างผลลัพธ์การจัดรูปแบบที่ถูกต้อง

    \item \textbf{สภาพแวดล้อม (Environments)} สภาพแวดล้อมใช้เพื่อกำหนดขอบเขตของเนื้อหาที่ต้องการให้มีรูปแบบเฉพาะ หรือมีพฤติกรรมพิเศษ มักเริ่มต้นด้วย \textbf{\textbackslash begin\{ชื่อสภาพแวดล้อม\}} และปิดท้ายด้วย \textbf{\textbackslash end\{ชื่อสภาพแวดล้อม\}} เช่น
    \begin{mycustomitem}
        \item \textbf{\textbackslash begin\{itemize\} ... \textbackslash end\{itemize\}} สำหรับสร้างรายการแบบจุด (bullet points)
        \item \textbf{\textbackslash begin\{enumerate\} ... \textbackslash end\{enumerate\}} สำหรับสร้างรายการแบบลำดับเลข
        \item \textbf{\textbackslash begin\{align\} ... \textbackslash end\{align\}} สำหรับจัดเรียงสมการคณิตศาสตร์หลายบรรทัด
        \item \textbf{\textbackslash begin\{figure\} ... \textbackslash end\{figure\}} สำหรับจัดการตำแหน่งของรูปภาพและใส่คำบรรยาย
    \end{mycustomitem}
    
เมื่อระบุสภาพแวดล้อม LaTeX จะจัดการการจัดเรียง จัดการเว้นวรรค และรายละเอียดอื่น ๆ ที่เกี่ยวข้องกับสภาพแวดล้อมนั้น ๆ โดยอัตโนมัติ

\end{mycustomitem}

\hspace*{1.5em} %ย่อหน้า
การทำงานแบบ Markup Language นี้ทำให้เอกสารต้นฉบับของ LaTeX เป็นไฟล์ข้อความธรรมดา (.tex) ซึ่งง่ายต่อการจัดการ คัดลอก หรือควบคุมเวอร์ชัน (Version Control) ร่วมกับผู้อื่น


\hspace*{1.5em} %ย่อหน้า
\textbf{ความแตกต่างจากโปรแกรมประมวลผลคำทั่วไป (เช่น Word Processor)}

\hspace*{1.5em} %ย่อหน้า
เพื่อความเข้าใจที่ชัดเจน มาดูความแตกต่างระหว่าง LaTeX กับโปรแกรมประมวลผลคำทั่วไปอย่าง Microsoft Word



\begin{table}[t]
    \caption{\fontSixTeen{การเปรียบเทียบระหว่าง LaTeX และโปรแกรมประมวลผลคำ (เช่น Word)}}
    \label{tab:latex-word-compare}
    \fontSixTeen 
    % เปลี่ยน X เป็น L สำหรับคอลัมน์ที่ 2 และ 3
    \begin{tabularx}{\textwidth}{|l|L|L|} 
        \hline
        % ใช้ \multicolumn เพื่อจัดกึ่งกลางเฉพาะหัวตาราง
        \textbf{คุณสมบัติ} & \multicolumn{1}{|>{\centering\arraybackslash}X|}{\textbf{LaTeX}} & \multicolumn{1}{|>{\centering\arraybackslash}X|}{\textbf{โปรแกรมประมวลผลคำ}} \\
        \hline
        % เนื้อหาตารางจะถูกจัดชิดซ้าย (left-aligned) ตามการกำหนดคอลัมน์ |l|L|L|
        \textbf{แนวคิดหลัก} & \textbf{Markup Language} เน้นการกำหนดโครงสร้างและเนื้อหาด้วยคำสั่ง (Content-Oriented) & \textbf{WYSIWYG} เน้นการจัดรูปแบบที่มองเห็นได้ทันที (Layout-Oriented) \\
        \hline
        \textbf{การทำงาน} & เขียนโค้ด (Source Code) $\rightarrow$ คอมไพล์ (Compile) $\rightarrow$ ได้ผลลัพธ์ (PDF/DVI) & พิมพ์และจัดรูปแบบได้ทันทีบนหน้าจอ \\
        \hline
        \textbf{การควบคุมรูปแบบ} & ควบคุมด้วยคำสั่งและแพ็กเกจ (Package) ระดับสูง (เชิงโครงสร้าง) & ควบคุมด้วยเมนู แถบเครื่องมือ และการคลิกเมาส์ (เชิงภาพ) \\
        \hline
        \textbf{ความแม่นยำ} & \textbf{สูงมาก} เหมาะสำหรับสมการคณิตศาสตร์ที่ซับซ้อน ตาราง การอ้างอิง บรรณานุกรม & ดีในระดับทั่วไป อาจซับซ้อนเมื่อเจอสมการหรือโครงสร้างที่ซับซ้อนมาก \\
        \hline
        \textbf{ความคงเส้นคงวา} & \textbf{สูง} การจัดรูปแบบเป็นมาตรฐาน ไม่เพี้ยนเมื่อเปิดต่างเครื่องหรือต่างระบบ & อาจเพี้ยนได้เมื่อเปิดต่างเวอร์ชันหรือต่างระบบปฏิบัติการ \\
        \hline
        \textbf{การเรียนรู้} & \textbf{สูง} ต้องเรียนรู้คำสั่งและแนวคิดใหม่ ๆ ใช้เวลาปรับตัว & \textbf{ต่ำ} ใช้งานง่าย เรียนรู้เร็วสำหรับงานทั่วไป \\
        \hline
        \textbf{ความยืดหยุ่น} & \textbf{สูง} มีแพ็กเกจเสริมมากมายสำหรับงานเฉพาะทาง สร้างเทมเพลตและสไตล์ได้เอง & ยืดหยุ่นพอสมควร มีเทมเพลตสำเร็จรูป แต่ปรับแต่งเชิงลึกได้จำกัด \\
        \hline
        \textbf{ไฟล์ต้นฉบับ} & \textbf{ข้อความธรรมดา (.tex)} จัดการด้วย Version Control ได้ดี & \textbf{ไบนารี (.docx)} ยากต่อการจัดการด้วย Version Control \\
        \hline
        \textbf{ค่าใช้จ่าย} & \textbf{ฟรีและ Open Source} & มักมีค่าใช้จ่าย (Licensed Software) \\
        \hline
        \textbf{กลุ่มผู้ใช้งานหลัก} & นักวิชาการ นักวิทยาศาสตร์ วิศวกร นักคณิตศาสตร์ นักพัฒนาซอฟต์แวร์ & ผู้ใช้ทั่วไป นักเรียน นักศึกษา สำนักงาน \\
        \hline
    \end{tabularx}
\end{table}

\hspace*{1.5em} %ย่อหน้า
\textbf{แนวคิดของการแยกเนื้อหาออกจากรูปแบบ (Content-Form Separation)}

\hspace*{1.5em} %ย่อหน้า
คือหัวใจสำคัญที่ทำให้ LaTeX แตกต่างและทรงพลัง

\begin{mycustomenum2}
    \item \textbf{เน้นเนื้อหา ไม่ใช่หน้าตา} ใน LaTeX จะมุ่งเน้นไปที่การเขียนเนื้อหาของเอกสารและกำหนดประเภทของเนื้อหานั้น ๆ (เช่น นี่คือหัวข้อหลัก นี่คือย่อหน้า นี่คือสมการ นี่คือรายการ) โดยไม่ต้องกังวลว่าข้อความเหล่านั้นจะถูกจัดวางบนหน้ากระดาษอย่างไร ตัวอย่างเช่น แทนที่จะคิดว่าข้อความนี้ต้องอยู่ตรงกลางและเป็นตัวหนาขนาด 16pt
    \item \textbf{ระบบจัดการรูปแบบอัตโนมัติ} เมื่อคอมไพล์เอกสาร LaTeX จะทำหน้าที่เป็นนักออกแบบที่ชาญฉลาด โดยตัดสินใจว่าข้อความแต่ละส่วนควรจะถูกจัดวางอย่างไรบนหน้ากระดาษ เพื่อให้ได้ผลลัพธ์ที่สวยงาม 
    \item \textbf{ประโยชน์จากการแยกส่วน}
    \begin{mycustomenum2}
        \item \textbf{ประหยัดเวลา} ผู้เขียนสามารถมุ่งเน้นไปที่การสร้างเนื้อหาที่มีคุณภาพ โดยไม่ต้องเสียเวลาไปกับการปรับแต่งรูปแบบเล็ก ๆ
        \item \textbf{ความคงเส้นคงวา} ทุกส่วนของเอกสารที่มีประเภทเดียวกัน (เช่น ทุกหัวข้อรอง) จะมีรูปแบบที่เหมือนกันตลอดทั้งเล่ม ทำให้เอกสารดูเป็นมืออาชีพและเป็นระเบียบ
        \item \textbf{ทำงานร่วมกันสะดวก} เมื่อมีหลายคนทำงานบนเอกสารเดียวกัน การแยกเนื้อหาออกจากรูปแบบจะช่วยลดความขัดแย้งในการแก้ไข (merge conflicts) เพราะแต่ละคนเน้นที่เนื้อหาของตนเอง
    \end{mycustomenum2}
\end{mycustomenum2}

\hspace*{1.5em} %ย่อหน้า
สรุปคือ สถาปัตยกรรมของ LaTeX ที่เน้นการทำงานแบบ Markup Language และการแยกเนื้อหาออกจากรูปแบบ ทำให้มันเป็นเครื่องมือที่ทรงพลังและมีประสิทธิภาพสูงสำหรับการสร้างเอกสารที่ซับซ้อน โดยเฉพาะอย่างยิ่งในแวดวงวิชาการและวิทยาศาสตร์ที่ต้องการความแม่นยำ ความคงเส้นคงวา และคุณภาพการเรียงพิมพ์ระดับสูง \parencite{overleaflearn}






%*******************************************
% งานวิจัยที่เกี่ยวข้อง
%*******************************************
\section{รายงานการค้นคว้า การศึกษา หรืองานวิจัยที่เกี่ยวข้อง}

\subsection{การใช้ LaTeX สำหรับสร้างเอกสารงานวิชาการ}
\hspace*{1.5em} %ย่อหน้า
ฑิตยา หวานวารี ได้จัดทำเอกสารเรื่อง \textbf{“การใช้ LaTeX สำหรับสร้างเอกสารงานวิชาการ”} ซึ่งเป็นแนวทางปฏิบัติที่ครอบคลุมสำหรับการใช้งาน LaTeX ในการจัดทำวิทยานิพนธ์ โดยเฉพาะอย่างยิ่งสำหรับภาษาไทยและภาษาอังกฤษภายในกรอบของจุฬาลงกรณ์มหาวิทยาลัย งานนี้ได้อธิบายพื้นฐานของ TeX และ LaTeX ตั้งแต่การติดตั้งไปจนถึงโครงสร้างเอกสารวิชาการ รวมถึงให้คำแนะนำในการแก้ไขปัญหาเฉพาะทางที่พบบ่อยเมื่อต้องทำงานกับข้อความภาษาไทย แสดงให้เห็นถึงศักยภาพของ LaTeX ในการสร้างเอกสารทางวิชาการที่ซับซ้อนและมีคุณภาพสูง \parencite{wanwaree2023}

\subsection{การพัฒนาการจัดทำเอกสารวิชาการด้วยภาษา TeX}
\hspace*{1.5em} %ย่อหน้า
ธานินทร์ คูพูลทรัพย์ ได้เผยแพร่บทความวิจัยเรื่อง \textbf{“การพัฒนาการจัดทำเอกสารวิชาการด้วยภาษา TeX”} ใน ThaiJo ซึ่งเป็นการนำเสนอผลการทดลองสร้างต้นแบบเอกสารวิชาการ เช่น รายงานวิจัยและบทความวิจัย โดยใช้ LaTeX และ XeTeX เพื่อให้ได้รูปแบบที่ตรงตามข้อกำหนดของคณะฯ งานวิจัยนี้ได้ระบุถึงประโยชน์ของการใช้แพ็กเกจมาตรฐานและแพ็กเกจเสริมที่จำเป็น รวมถึงปัญหาที่พบระหว่างการพัฒนา เช่น การอ้างอิงชื่อผู้แต่งภาษาไทย และแนวทางแก้ไขที่ใช้ ซึ่งเป็นข้อมูลที่มีค่าสำหรับนักวิจัยและสถาบันการศึกษาที่ต้องการนำ LaTeX ไปประยุกต์ใช้ในการจัดทำเอกสารวิชาการของตน \parencite{lampangrajabhat2013}

\subsection{การออกแบบรูปแบบเอกสารสำหรับปริญญานิพนธ์ด้วยโปรแกรมสำเร็จรูป LaTex ภาษาไทย}
\hspace*{1.5em} %ย่อหน้า
ธันยา ภู่วโรดม ได้นำเสนอเนื้อหาใน \textbf{“การออกแบบรูปแบบเอกสารสำหรับปริญญานิพนธ์ด้วยโปรแกรมสำเร็จรูป LaTex ภาษาไทย”} ซึ่งเป็นส่วนหนึ่งของเอกสารการเรียนการสอนหรือโครงงาน โดยให้ภาพรวมเกี่ยวกับที่มาและความสำคัญของ TeX และ LaTeX เน้นย้ำข้อดีของการเป็นเครื่องมือเรียงพิมพ์ที่ให้คุณภาพสูง เหมาะสำหรับงานวิทยาศาสตร์และคณิตศาสตร์ และการที่ไฟล์ต้นฉบับเป็นข้อความธรรมดาทำให้ง่ายต่อการจัดการ นอกจากนี้ยังชี้ให้เห็นถึงความท้าทายในการใช้ LaTeX กับภาษาไทย โดยเฉพาะประเด็นการตัดคำและการตั้งค่าฟอนต์ ซึ่งเป็นสิ่งสำคัญที่ต้องพิจารณาเมื่อนำ LaTeX มาใช้ในบริบทของภาษาไทย \parencite{Tanya2008}
